\chapter{Introdução}

Escrever bem é uma arte que exige muita técnica e dedicação. Há vários bons livros
sobre como escrever uma boa dissertação ou tese. Um dos trabalhos pioneiros e mais
conhecidos nesse sentido é o livro de Umberto Eco \cite{eco:09} intitulado 
\textit{Como se faz uma tese}; é uma leitura bem interessante mas, como foi escrito 
em 1977 e é voltado para teses de graduação na Itália, não se aplica tanto a nós.

Para a escrita de textos em Ciência da Computação, o livro de Justin Zobel, 
\textit{Writing for Computer Science} \cite{zobel:04} é uma leitura obrigatória. 
O livro \textit{Metodologia de Pesquisa para Ciência da Computação} de 
Raul Sidnei Wazlawick \cite{waz:09} também merece uma boa lida.
Já para a área de Matemática, dois livros recomendados são o de Nicholas Higham,
\textit{Handbook of Writing for Mathematical Sciences} \cite{Higham:98} e o do criador
do \TeX, Donald Knuth, juntamente com Tracy Larrabee e Paul Roberts, 
\textit{Mathematical Writing} \cite{Knuth:96}.

O uso desnecessário de termos em lingua estrangeira deve ser evitado. No entanto,
quando isso for necessário, os termos devem aparecer \emph{em itálico}.

\begin{footnotesize}
\begin{verbatim}
Modos de citação:
indesejável: [AF83] introduziu o algoritmo ótimo.
indesejável: (Andrew e Foster, 1983) introduziram o algoritmo ótimo.
certo : Andrew e Foster introduziram o algoritmo ótimo [AF83].
certo : Andrew e Foster introduziram o algoritmo ótimo (Andrew e Foster, 1983).
certo : Andrew e Foster (1983) introduziram o algoritmo ótimo.
\end{verbatim}
\end{footnotesize}

Uma prática recomendável na escrita de textos é descrever as legendas das
figuras e tabelas em forma auto-contida: as legendas devem ser razoavelmente
completas, de modo que o leitor possa entender a figura sem ler o texto onde a
figura ou tabela é citada.  

Apresentar os resultados de forma simples, clara e completa é uma tarefa que
requer inspiração. Nesse sentido, o livro de Edward Tufte \cite{tufte01:visualDisplay},
\textit{The Visual Display of Quantitative Information}, serve de ajuda na 
criação de figuras que permitam entender e interpretar dados/resultados de forma
eficiente.

\section{Considerações Preliminares}

\paragraph{}
Lorem ipsum dolor sit amet, consectetur adipiscing elit, sed do eiusmod tempor incididunt ut labore et dolore magna aliqua. Eu volutpat odio facilisis mauris sit. Semper auctor neque vitae tempus. Pellentesque massa placerat duis ultricies lacus sed. Aliquam etiam erat velit scelerisque in dictum non consectetur.

\section{Objetivos}

\paragraph{}
Congue nisi vitae suscipit tellus mauris a diam. Aliquet sagittis id consectetur purus. Nullam non nisi est sit amet. Vestibulum mattis ullamcorper velit sed ullamcorper morbi tincidunt ornare. Id diam vel quam elementum pulvinar. Bibendum neque egestas congue quisque egestas diam in arcu. In vitae turpis massa sed elementum tempus egestas sed. Cursus euismod quis viverra nibh cras pulvinar mattis nunc. Aliquet porttitor lacus luctus accumsan tortor. Enim lobortis scelerisque fermentum dui.

\section{Contribuições}

As principais contribuições deste trabalho são as seguintes:

\begin{itemize}
    \item Item 1: Lacus vestibulum sed arcu non odio euismod lacinia. Nulla posuere sollicitudin aliquam ultrices sagittis. Aliquam ut porttitor leo a diam. Elit at imperdiet dui accumsan sit amet nulla facilisi morbi. Parturient montes nascetur ridiculus mus mauris vitae ultricies leo.
    \item Item 2: Nulla pellentesque dignissim enim sit amet venenatis urna cursus. Id eu nisl nunc mi ipsum faucibus. Quisque id diam vel quam elementum pulvinar etiam non. Quis lectus nulla at volutpat diam ut venenatis tellus in. Sed euismod nisi porta lorem mollis aliquam ut. Dui nunc mattis enim ut. Nulla pharetra diam sit amet nisl suscipit adipiscing.
\end{itemize}